\chapter*{}

\begin{titlepage}
 
 
\setlength{\centeroffset}{-0.5\oddsidemargin}
\addtolength{\centeroffset}{0.5\evensidemargin}
\thispagestyle{empty}

\noindent\hspace*{\centeroffset}\begin{minipage}{\textwidth}

\centering

\vspace{3.3cm}

%si el proyecto tiene logo poner aquí
%\includegraphics{imagenes/logo.png} 
\vspace{0.5cm}

% Title

{\Huge\bfseries unsersampling\\
}
\noindent\rule[-1ex]{\textwidth}{3pt}\\[3.5ex]
{\large\bfseries Una biblioteca en Scala para clasificación no balanceada.\\[4cm]}
\end{minipage}

\vspace{2.5cm}
\noindent\hspace*{\centeroffset}\begin{minipage}{\textwidth}
\centering

\textbf{Autor}\\ {Néstor Rodríguez Vico}\\[2.5ex]
\textbf{Directores}\\
{Alberto Fernandez Hilario\\
Salvador García Lopez}\\[2cm]
\end{minipage}
\vspace{\stretch{2}}

 
\end{titlepage}




\cleardoublepage
\thispagestyle{empty}

\begin{center}
{\large\bfseries undersampling: una biblioteca en Scala para \textit{undersampling} en clasificación no balanceada.}\\
\end{center}
\begin{center}
Néstor Rodríguez Vico\\
\end{center}

\noindent{\textbf{Palabras clave:} clasificación no balanceada, preprocesamiento, Scala, reducción de datos.}\\

\vspace{0.7cm}
\noindent{\textbf{Resumen}}\\

En este trabajo se ha implementado una biblioteca en Scala para solventar el problema del aprendizaje en conjuntos de datos no balanceados. Se han implementado 15 algoritmos de preprocesamiento de datos, los cuales incluyen algoritmos clásicos y algoritmos con un enfoque más modernos. Dichos algoritmos han sido ejecutados sobre más de 20 conjuntos distintos de datos para probar su calidad y poder compararlos posteriormente usando dos clasificadores distintos, un árbol de decisión \textit{C4.5} y una máquina de soporte vectorial.
\cleardoublepage

\thispagestyle{empty}

\begin{center}
{\large\bfseries undersampling: a Scala library for \textit{undersampling} in imbalanced classification.}\\
\end{center}
\begin{center}
Néstor Rodríguez Vico\\
\end{center}

\noindent{\textbf{Keywords:} imbalanced classification, preprocess, Scala, data reduction.}\\

\vspace{0.7cm}
\noindent{\textbf{Abstract}}\\

In this project, a Scala library has been implemented to solve the learning problem in imbalanced data sets. 15 data preprocessing algorithms have been implemented, including classic algorithms and algorithms with a modern approach. These algorithms have been executed in more than 20 data sets to prove their quality and be compared using two different classifiers, a \textit{C4.5} tree classifier and a support vector machine.

\chapter*{}
\thispagestyle{empty}

\noindent\rule[-1ex]{\textwidth}{2pt}\\[4.5ex]

Yo, \textbf{Néstor Rodríguez Vico}, alumno de la titulación Ingeniería Informática de la \textbf{Escuela Técnica Superior de Ingenierías Informática y de Telecomunicación de la Universidad de Granada}, con DNI 75573052C, autorizo la ubicación de la siguiente copia de mi Trabajo Fin de Grado en la biblioteca del centro para que pueda ser consultada por las personas que lo deseen.

\vspace{6cm}

\noindent Fdo: Néstor Rodríguez Vico

\vspace{2cm}

\begin{flushright}
Granada, \today
\end{flushright}


\chapter*{}
\thispagestyle{empty}

\noindent\rule[-1ex]{\textwidth}{2pt}\\[4.5ex]

D. \textbf{Alberto Fernández Hilario}, Profesor del Área de \textit{Soft Computing and Intelligent Information Systems} del Departamento de Computación y Sistemas Inteligentes de la Universidad de Granada.

\vspace{0.5cm}

D. \textbf{Salvador García López}, Profesor del Área de \textit{Soft Computing and Intelligent Information Systems} del Departamento de Computación y Sistemas Inteligentes de la Universidad de Granada.


\vspace{0.5cm}

\textbf{Informan:}

\vspace{0.5cm}

Que el presente trabajo, titulado \textit{\textbf{undersampling, una biblioteca en Scala para \textit{undersampling} en clasificación no balanceada}}, ha sido realizado bajo su supervisión por \textbf{Néstor Rodríguez Vico}, y autorizamos la defensa de dicho trabajo ante el tribunal que corresponda.

\vspace{0.5cm}

Y para que conste, expiden y firman el presente informe en Granada a \today.

\vspace{1cm}

\textbf{Los directores:}

\vspace{5cm}

\noindent \textbf{Alberto Fernández Hilario \ \ \ \ \ Salvador García López}

\chapter*{Agradecimientos}
\thispagestyle{empty}

       \vspace{1cm}


Muchas gracias a mi familia por el apoyo incondicional en todos estos años de carrera, especialmente en este último y duro año. A mis tutores, profesores y amigos que me han ayudado a llegar a donde estoy a día de hoy. \\

Y a Saray, por estar siempre ahí.